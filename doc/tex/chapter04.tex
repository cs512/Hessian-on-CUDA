%%==================================================
%% chapter04.tex for SJTU Master Thesis
%% Encoding: UTF-8
%%==================================================

\chapter{算法在CUDA平台的实现}
  本实习在Ubuntu 14.04.1 LTS上实现了Hessian-Affine的CUDA版本,并且通过不同分辨率的图片测试出了算法中各个函数的加速比,以及整体的加速比。
  \subparagraph{测试数据}
    由于本算法会被两个变量影响性能:
    \begin{enumerate}
      \item 图像大小
      \item 图像复杂度
    \end{enumerate}
    \par
    因此,为了测试加速比,我选用了10张分辨率为1024*768的图像作测试,由于以特征点数量作为因变量,时间作为参变量。同时又选取了同一张图片的10种分辨率,在测试图像大小对算法的影响。以上图片均来自Oxford 5k数据集。
  \subparagraph{数据流}
    本算法的数据流简要如下:生成尺度空间->在尺度空间中计算每个点的Hessian相应->精确定位特征点->寻找仿射标准形->计算SIFT描述子。在这个算法中,每个部分的都可以并行,但是并行的粒度不同,导致了CUDA的加速比不同。
  \subparagraph{生成尺度空间}
    这部分中,由于尺度空间的生成用到了大量的逻辑控制运算,在编写这部分代码时,逻辑运算在CPU侧运行,数据与代码运算在GPU侧运行。在这部分算法中,关键的函数有\fbox{\tt helper.cpp::halfImage}、\fbox{\tt helper.cpp::doubleImage}与\fbox{\tt helper.cpp::gaussianBlur}。这三个函数的加速比如下:


