%%==========================
%% chapter01.tex for SJTU Master Thesis
%% based on CASthesis
%% modified by wei.jianwen@gmail.com
%% version: 0.3a
%% Encoding: UTF-8
%% last update: Dec 5th, 2010
%%==================================================

%\bibliographystyle{sjtu2} %[此处用于每章都生产参考文献]
\chapter{综述}
\label{chap:intro}

  \section{背景}
  \label{sec:background}
    随着社会的发展,计算机科学的进步,人们身边充斥了各种各样的信息。而视觉作为人类最重要的感官,视觉信息占据了我们身边信息的主要部分,大量的视觉信息通过人眼被感知。通过这个过程,我们得以了解世界、学习知识。网络技术的发展,让我们身边的信息越来越丰富、内容越来越杂乱,让我们很难一眼看出其中蕴含的信息。并且由于信息爆炸性增长,越来越不可能使用纯人工手段进行信息处理,所以,我们需要通过计算机来帮助我们识别与处理信息。
    \par
    在数字图像处理领域,为了识别一张图像,通常使用兴趣点作为识别的基础。兴趣点通常指一张图像中,包含信息较多的点。由于兴趣点及其重要,并且经常使用,检测兴趣点的算法在需要有一定的效率,否则无法处理极大的数据量。除了算法本身的速度外,GPU、FPGA等芯片可以提供比CPU更加强大计算能力,并且由于算法本身特性可能更适合在GPU等芯片的体系结构下运行,在GPU上可以获得更高的效率。将这类算法移植在GPU上是有必要的。

  \section{任务目标}
  \label{sec:aim}
    本研究基于Hessian-Affine兴趣点检测算法与nVidia的GPU平台及其编程语言CUDA。兴趣点检测算法基于尺度不变兴趣点检测算法与仿射不变兴趣点检测算法,尺度不变是指同一个兴趣点,在不同的尺度下都可以被检测出,并且给出一个统一的尺度信息。仿射不变是指兴趣点经过仿射变换之后(例如视角的移动),算法可以检测出兴趣点。在Hessian Affine兴趣点检测算法中,使用Hessian-Laplace 检测子检测位置和尺度,附近的仿射由Affine Adaptation 程序决定。
    \par
    本算法原本由Perdoch与2009年编写,并开源在\href{https://github.com/perdoch/hesaff}{Github}中,本次任务的目标是将其移植到NVIDIA GTX600系列显卡中,并计算加速比。
    \par
    CUDA语言基于C语言,可以编译出GPU原生代码。在执行时,将运算数据复制到GPU内存中,再通过指令驱动GPU,然后GPU通过核心级的并行处理数据,最后将结果传回主存中。由于GPU的体系结构与CPU区别很大,GPU核心很多,但是单核心性能较低,并且GPU的数学运算能力更强、逻辑运算能力很弱,如何优化算法将其适合GPU是本研究的重点。
    \par
    nVidia的GPU分为Tesla、Fermi、Kepler、Maxwell等架构,不同架构对CUDA的支持不尽相同,在Tesla架构中,每个SM包含8个流处理器(SP),有三级存储结构:寄存器、共享内存、全局内存,速度从高到低。由于并行的体系结构,并且在GPU中访问全局内存过于缓慢,本研究会专注于算法数据流的设计,降低I/O开销,从而降低计算时间。
